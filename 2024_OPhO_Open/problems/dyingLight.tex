\begin{problem}{\textbf{\textsc{Tắt đèn}}} 

Follin tạo ra một bể chứa một loại chất lỏng đặc biệt với chỉ số khúc xạ $n = 1 + i(1\cdot 10^{-6}).$ Có vẻ hơi phức tạp. Khi làm việc với chất lỏng này, anh ta vô tình làm rơi một cảm biến ánh sáng vào trong bể. Follin chiếu một tia laser đỏ với bước sóng $\lambda=700\;\text{nm}$ và cường độ trong chân không $I_0 = 5\cdot 10^{6}\;\mathrm{W/m^2}$ xuống chất lỏng. Cảm biến ánh sáng chìm xuống bao xa trước khi phát hiện được cường độ nhỏ hơn $I_f = 10^{-10}\;\mathrm{W/m^2}$? Giả sử rằng phòng thí nghiệm hoàn toàn tối và ánh sáng laser được truyền hoàn toàn vào chất lỏng.  

\end{problem}
