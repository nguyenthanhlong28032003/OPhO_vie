\begin{problem}{\textbf{\textsc{Peculiar Pump}}} A box with volume $V=1\;\mathrm{m}^3$ and initial temperature $T_0=100\;\mathrm{K}$ contains monatomic ideal gas initially kept at a constant low pressure $p_1=100\;\mathrm{Pa}$. The mass of the gas molecules is $m=7\times10^{-27}\;\mathrm{kg}$. The box is connected to a very large reservoir which contains monatomic ideal gas at temperature $T_2=400\;\mathrm{K}$ and pressure $p_2>p_1$. A small hole with area $A=10^{-8}\;\mathrm{m}^2$ is poked in the box at time $t=0\;\mathrm{days}$, allowing gas to escape. (The box is no longer at constant pressure after this point.) For every molecule of gas that escapes through this hole, 3 molecules of gas are let into the box from the reservoir. (The particles let in from the reservoir are selected with probability proportional to their velocity component perpendicular to the hole. In other words, the particles effuse through the hole but with the door to the hole restricting their number to 3 times the number of outgoing particles.) How many days does it take for the temperature in the box to double? Assume changes in the reservoir's pressure and temperature are negligible. For reference, the rate of particles escaping out of a hole with area A is given by:\begin{equation*}
    \Phi=\frac{pA}{\sqrt{2\pi mk_BT}}
\end{equation*}
where $p$ is the pressure of the gas, $m$ is the mass of the gas molecules, $k_B$ is the Boltzmann constant and $T$ is the temperature of the gas. Please note that the average energy per particle of particles leaving via effusion is $2kT$.

\end{problem}