%\begin{problem}{\textbf{\textsc{Peculiar Pump}}} A box with volume $V=1\;\mathrm{m}^3$ and initial temperature $T_0=100\;\mathrm{K}$ contains monatomic ideal gas initially kept at a constant low pressure $p_1=100\;\mathrm{Pa}$. The mass of the gas molecules is $m=7\times10^{-27}\;\mathrm{kg}$. The box is connected to a very large reservoir which contains monatomic ideal gas at temperature $T_2=400\;\mathrm{K}$ and pressure $p_2>p_1$. A small hole with area $A=10^{-8}\;\mathrm{m}^2$ is poked in the box at time $t=0\;\mathrm{days}$, allowing gas to escape. (The box is no longer at constant pressure after this point.) For every molecule of gas that escapes through this hole, 3 molecules of gas are let into the box from the reservoir. (The particles let in from the reservoir are selected with probability proportional to their velocity component perpendicular to the hole. In other words, the particles effuse through the hole but with the door to the hole restricting their number to 3 times the number of outgoing particles.) How many days does it take for the temperature in the box to double? Assume changes in the reservoir's pressure and temperature are negligible. For reference, the rate of particles escaping out of a hole with area A is given by:\begin{equation*}
		%\Phi=\frac{pA}{\sqrt{2\pi mk_BT}}
	%\end{equation*}
	%where $p$ is the pressure of the gas, $m$ is the mass of the gas molecules, $k_B$ is the Boltzmann constant and $T$ is the temperature of the gas. Please note that the average energy per particle of particles leaving via effusion is $2kT$.
	
%\end{problem}

\begin{problem}{\textbf{\textsc{Peculiar Pump}}} Một hộp có thể tích $V=1\;\mathrm{m}^3$ và nhiệt độ ban đầu $T_0=100\;\mathrm{K}$ chứa khí lý tưởng đơn nguyên tử được giữ ở áp suất thấp không đổi $p_1=100\;\mathrm{Pa}$. Khối lượng của phân tử khí là $m=7\times10^{-27}\;\mathrm{kg}$. Hộp được nối với một bể chứa rất lớn có chứa khí lý tưởng đơn nguyên tử ở nhiệt độ $T_2=400\;\mathrm{K}$ và áp suất $p_2>p_1$. Một lỗ nhỏ với diện tích $A=10^{-8}\;\mathrm{m}^2$ được khoan trên hộp vào thời điểm $t=0\;\mathrm{days}$, cho phép khí thoát ra. (Hộp không còn ở áp suất không đổi sau thời điểm này.) Với mỗi phân tử khí thoát ra qua lỗ này, có 3 phân tử khí từ khoang chứa được đưa vào hộp. (Các phân tử được đưa vào từ khoang chứa được chọn với xác suất tỷ lệ thuận với thành phần vận tốc của chúng vuông góc với lỗ. Nói cách khác, các phân tử khuếch tán qua lỗ nhưng cửa lỗ giới hạn số lượng của chúng là 3 lần số phân tử thoát ra ngoài.) Mất bao nhiêu ngày để nhiệt độ trong hộp tăng gấp đôi? Giả sử sự thay đổi áp suất và nhiệt độ trong khoang chứa là không đáng kể. Tham khảo, tốc độ phân tử thoát ra ngoài qua một lỗ có diện tích A được tính bởi:\begin{equation*}
    \Phi=\frac{pA}{\sqrt{2\pi mk_BT}}
\end{equation*}
trong đó $p$ là áp suất của khí, $m$ là khối lượng của phân tử khí, $k_B$ là hằng số Boltzmann và $T$ là nhiệt độ của khí. Xin lưu ý rằng năng lượng trung bình trên mỗi phân tử của các phân tử thoát ra qua quá trình khuếch tán là $2kT$.

\end{problem}