\begin{problem}
{\textbf{\textsc{Đường vòng 1}}} Một dốc có chiều dài $d$ được nâng lên một góc $\theta$ ($0^{\circ} < \theta < 90^{\circ}$) so với mặt phẳng ngang. Một khối có khối lượng $m$ được đặt ở đỉnh của dốc, với hệ số ma sát giữa khối và dốc là $\mu$. Khi khối di chuyển đến đáy của dốc, nó giữ nguyên vận tốc khi được chuyển tiếp mượt mà lên một đường tròn không ma sát với bán kính $d$ và góc nghiêng $\theta$, quay trên đường tròn mà không bị trượt ra ngoài. Một \emph{solution} là một tập hợp các giá trị $\{d, \mu, \theta\}$ dẫn đến tình huống đã mô tả ở trên. Giá trị lớn nhất của $\theta$ (tính bằng độ) mà vẫn tồn tại giải pháp là gì? 

\end{problem}
