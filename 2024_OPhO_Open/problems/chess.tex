%\begin{problem}{\textbf{\textsc{Chess}}}
%Proposed rewording (by Eppu):
%Imagine a $4 \times 4$ grid with a particle in each cell. These particles move in an L-shape, similar to knights in chess. Every second, a particle moves randomly in one of the cells it can access with an L jump. Two or more particles are allowed to occupy the same cell. After some time, this system will reach an equilibrium. If the (statistical) temperature of the system and thus each cell is $T=1\; \mathrm{mK}$, all the cells will have a definite energy level. Find the difference between the highest and lowest energy states in $\mathrm{eV}$.

%Imagine a $4 \times 4$ lattice with a particle in each cell. These particles move in an L-shape,
%similar to knights in chess. Every second, a particle moves randomly in one of the cells it
%can access with an L jump. Two or more particles are allowed to occupy the same cell. After some time, this system will reach an equilibrium. 

% Let $A,B,C,D$ be the expected number of particles in each cell of the second row.  Find $p+q$, where $p,q$ are relatively prime positive integers such that $\frac{p}{q}=A\cdot B\cdot C\cdot D$.

%We can assign an energy to each cell based on the probability of a particle begin there. Consider $T = 1\;\mathrm{mK}$ given. Find the difference between the smallest and the largest equivalent energies in $eV$.

%For a harder problem, try it on an $n \times n$ lattice. If this is easy too switch to queen-like moves and try get a closed form formula. (may be a bit more mathematical)



%\end{problem}

\begin{problem}{\textbf{\textsc{Chess}}}
	%Proposed rewording (by Eppu):
	Hãy tưởng tượng một lưới $4 \times 4$ với một hạt trong mỗi ô. Các hạt này di chuyển theo hình dạng chữ L, tương tự như các quân mã trong cờ vua. Mỗi giây, một hạt di chuyển ngẫu nhiên đến một trong các ô mà nó có thể tiếp cận bằng một bước nhảy hình L. Hai hay nhiều hạt có thể chiếm cùng một ô. Sau một thời gian, hệ thống này sẽ đạt đến trạng thái cân bằng. Nếu nhiệt độ (thống kê) của hệ và do đó của mỗi ô là $T=1\; \mathrm{mK}$, tất cả các ô sẽ có mức năng lượng xác định. Hãy tìm độ chênh lệch giữa trạng thái năng lượng cao nhất và thấp nhất bằng đơn vị $\mathrm{eV}$.
	
	%Imagine a $4 \times 4$ lattice with a particle in each cell. These particles move in an L-shape,
	%similar to knights in chess. Every second, a particle moves randomly in one of the cells it
	%can access with an L jump. Two or more particles are allowed to occupy the same cell. After some time, this system will reach an equilibrium. 
	
	% Let $A,B,C,D$ be the expected number of particles in each cell of the second row.  Find $p+q$, where $p,q$ are relatively prime positive integers such that $\frac{p}{q}=A\cdot B\cdot C\cdot D$.
	
	%We can assign an energy to each cell based on the probability of a particle begin there. Consider $T = 1\;\mathrm{mK}$ given. Find the difference between the smallest and the largest equivalent energies in $eV$.
	
	%For a harder problem, try it on an $n \times n$ lattice. If this is easy too switch to queen-like moves and try get a closed form formula. (may be a bit more mathematical)
	
	
	
\end{problem}