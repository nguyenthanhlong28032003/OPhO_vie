\begin{problem}{\textbf{\textsc{Chess}}}
%Proposed rewording (by Eppu):
Imagine a $4 \times 4$ grid with a particle in each cell. These particles move in an L-shape, similar to knights in chess. Every second, a particle moves randomly in one of the cells it can access with an L jump. Two or more particles are allowed to occupy the same cell. After some time, this system will reach an equilibrium. If the (statistical) temperature of the system and thus each cell is $T=1\; \mathrm{mK}$, all the cells will have a definite energy level. Find the difference between the highest and lowest energy states in $\mathrm{eV}$.

%Imagine a $4 \times 4$ lattice with a particle in each cell. These particles move in an L-shape,
%similar to knights in chess. Every second, a particle moves randomly in one of the cells it
%can access with an L jump. Two or more particles are allowed to occupy the same cell. After some time, this system will reach an equilibrium. 

% Let $A,B,C,D$ be the expected number of particles in each cell of the second row.  Find $p+q$, where $p,q$ are relatively prime positive integers such that $\frac{p}{q}=A\cdot B\cdot C\cdot D$.

%We can assign an energy to each cell based on the probability of a particle begin there. Consider $T = 1\;\mathrm{mK}$ given. Find the difference between the smallest and the largest equivalent energies in $eV$.

%For a harder problem, try it on an $n \times n$ lattice. If this is easy too switch to queen-like moves and try get a closed form formula. (may be a bit more mathematical)



\end{problem}