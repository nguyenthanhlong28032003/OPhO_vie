\begin{solution}
    Lộ trình tối ưu là tăng tốc khối lượng đến vận tốc $v_{max}$, sau đó khi nó tiếp cận đến $\theta=90^{\circ}$, để động năng của khối lượng đưa nó lên đỉnh trong khi động cơ không cung cấp đủ mô men xoắn để chống lại trọng lực. 

    Ta sẽ xấp xỉ đáp án. Thiết lập của bài này tương tự như bài Motorized Pendulum 1, vì vậy đáp án sẽ là một giá trị nhỏ sai lệch từ đáp án của bài đó. Cụ thể, đặt $\tau_0 = 2mgl-\epsilon $. Gần chín mưới độ, mô men do trọng lực sinh ra xấp xỉ là $mgl$, và mô men xoắn của động cơ xấp xỉ $\tau_0 \frac{1+\delta}{2}$, với $\delta=90^{\circ} - \theta$ là khoảng cách đến đỉnh.

   Chúng ta có thể tìm điểm mà trọng lực bắt đầu thắng thế động cơ bằng cách đặt $mgl$ bằng $\tau_0 \frac{1+\delta}{2}$. Điều này xảy ra tại
    
    $$\delta=\epsilon/\tau_0 \approx \epsilon/2mgl$$
    
   Từ điểm này trở đi, công do trọng lực thực hiện trừ công do động cơ thực hiện sẽ dương, khiến khối lượng giảm tốc. Tuy nhiên, công này không được lớn hơn động năng của khối lượng, $\frac{1}{2}mv^2$. Xấp xỉ các mô men từ trọng lực và động cơ là các đường thẳng, mô men ròng cũng là một đường thẳng, vì vậy ta có thể tính công thực hiện để giảm tốc khối lượng từ điểm mà trọng lực hơn động cơ đến khi đạt chín mươi độ. Mô men xoắn ròng là 

    $$mgl-\tau_0 \frac{1+\delta}{2}\approx mgl-2mgl \frac{1+\delta}{2}=-mgl \delta $$

    Diện tích dưới đồ thị (tức là công ròng) từ $\delta=\epsilon/2mgl$ đến không là 

    $$\frac{1}{2} \cdot \epsilon/2mgl \cdot \epsilon/2 = \frac{\epsilon^2}{8mgl} $$

   Chúng ta có thể giải $\epsilon$ bằng cách đặt diện tích này bằng $\frac{1}{2} mv^2$, dẫn đến $\epsilon\approx6.6\ Nm$. Do đó, đáp án với ba chữ số có nghĩa là $\boxed{\tau_0=147.1-6.6=140.5 \;\mathrm{N\cdot m}}$.
    

\end{solution}
