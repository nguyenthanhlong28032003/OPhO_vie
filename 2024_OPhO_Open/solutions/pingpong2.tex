\begin{solution}
We can generalise the argument of the last problem by noting that
\[v_n\geq\sqrt{gh+g\sqrt{h^2+x^2}},\]
where $x$ is the distance of the last bounce from the net. Thus we wish to get the ball as close to the net as possible in $n$ bounces. This happens when all the previous bouncers happen at an angle $\ang{45}$. The range of one of these bounces is $\ell=\frac{v^2}{g}$ as one can find from the trajectory equation. Thus we have that the distance of the final bounce from the net is:
\[x=d-n\ell.\]
We also note that $v_n\geq \sqrt{2gh}$ based on the lower limit of the previous equation ($x=0$). So if we reach the net in less than $n$ bounces and thus get to $x=0$ for the last bounce, the minimal speed will still be $v_n$. I.e. when $d-n\ell < 0 \implies n>\frac{d}{2h}$ the minimal speed will be
\[v_n=\sqrt{2gh}.\]
As $d/2h\approx 4.5$, $N=5.$\\

Now if $n<N$, we won't reach the net in $n$ bounces, and the distance from the net for the last bounce is $x=d-n\ell.$ I.e. the speed to get the ball's last bounce to be a distance $x$ away will be $v=\sqrt{g(d-x)/n}.$ Similarly to the previous problems, this means that the optimal speed is achieved at the $x$ for which the optimal speed to get there is the same as to go over the net from there. I.e. we get that
\[h+\sqrt{h^2+x^2}=\frac{d-x}{n},\]
which yields the quadratic equation
\[(n^2-1)x^2+2(d-nh)x+2ndh-d^2=0,\]
from which we get
\[x=\frac{nh-d\pm n\sqrt{h^2+d^2-2ndh}}{n^2-1}.\]
As $d>2nh$ the smaller root is negative and thus non-physical. Hence, we take the positive root. Substituting this in one of the expressions for $v$ we get
\[v_n=\sqrt{\frac{g}{n^2-1}\left(nd-h-\sqrt{h^2+d^2-2ndh}\right)}.\]

Thus
\[v_{N-1}^N=v_4^5\approx\boxed{17.9\;(\mathrm{m/s})^5.}\]
\end{solution}