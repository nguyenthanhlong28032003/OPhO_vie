\begin{solution}
Giả sử neutron đến có hệ số Lorentz là $\gamma$. Lưu ý rằng năng lượng tổng của hệ thống là $E =(\gamma m_n + m_\alpha)c^2$ và động lượng ban đầu thỏa mãn $p^2 = (\gamma^2 - 1)m_n^2 c^2$. Năng lượng tổng được tối thiểu hóa nếu các proton và neutron cuối cùng đều có cùng một vận tốc, trong trường hợp này, khối lượng cuối cùng của hệ thống là $M=2m_p + 3m_n$. Do đó, theo định luật bảo toàn năng lượng và động lượng:
\begin{align*}E^2 &= p^2 c^2 + M^2 c^4\\ (\gamma^2 m_n^2 + 2\gamma m_n m_\alpha + m_\alpha ^2)c^4 &= (\gamma^2 - 1)m_n^2 c^4 + M^2 c^4\\ \gamma &= \frac{M^2 - m_n^2 - m_\alpha^2}{2m_n m_\alpha} = 1.0378\end{align*}
Ta tim được $v/c = \sqrt{1 - 1/\gamma^2} = \boxed{0.267}$.
\end{solution}