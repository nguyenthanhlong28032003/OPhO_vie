\begin{solution}
Sử dụng thế trung tâm, thế hiệu dụng của hạt được cho bởi

$$V(r) = \frac{L^2}{2mr^2}$$

Đối với một độ lệch nhỏ $\delta r$ chúng ta có thể thiết lập như sau:

$$V(d + \delta r) \approx V(d) + \frac{\cos^2\theta}{2}\bigg(\frac{d^2V}{dr^2}\bigg|_{r=d}\bigg)\delta r^2$$

$$\frac{d^2V}{dr^2} = \frac{3L^2}{mr^4}$$

$$\frac{mv^2}{d}\cos\theta = mg\sin\theta\longrightarrow v = \sqrt{gd\tan\theta}$$

$$L = mvd = md\sqrt{gd\tan\theta}$$

$$\frac{d^2V}{dr^2} = \frac{3mg\tan\theta}{d}$$

$$V(d+\delta r) \approx V(d) + \frac{1}{2}\frac{3mg\sin\theta\cos\theta}{d}\delta r^2$$

Từ đây, chúng ta thu được

$$\omega = \sqrt{\frac{3g\sin\theta\cos\theta}{d}}$$

$$T = \frac{2\pi}{\omega} = 2\pi\sqrt{\frac{2d}{3g\sin2\theta}}$$

hời gian này đạt giá trị tối thiểu khi $\theta = 45^{\circ}$, do đó, thay số vào cho kết quả $T_{min} \approx \boxed{3.664\;\mathrm{s}}$

\end{solution}