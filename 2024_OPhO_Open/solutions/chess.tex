\newpage
\begin{solution}
If we reach a steady state, we can assign a probability $p(x)$ to each cell $x$.
We can think of probability jumping, instead of the ball jumping.
So at one step, all the probability in $x$ disappears, and we need some probability jump back into it.\\

The probability jumping into $x$ is:
\begin{equation*}
    p(x) = \sum_{u \in \text{neighbours(x)}} p(u)p(u \rightarrow x)   
\end{equation*}
Now $p(u\rightarrow x)$ is just $\frac{1}{\text{deg}(u)}$. The degree of a cell, denoted $\text{deg}$ is the number
of other cells that can reach it
\begin{equation*}
    p(x) = \sum_{u \in \text{neighbours(x)}} \frac{p(u)}{\text{deg}(u)}  
\end{equation*}

Here, we can see that if the system reaches a steady state, $p(x)$ will be proportional to $\text{deg}(x)$.
So the solution is 
\begin{equation*}
    p(x) = \frac{\text{deg}(x)}{\sum_u \text{deg}(u)}
\end{equation*}

Which we can check in practice (with something like Mathematica), and it works.
\\

For our $4 \times 4$ grid, the degrees of the cells are:

\[
\begin{matrix}
2 & 3 & 3 & 2 \\
3 & 4 & 4 & 3 \\
3 & 4 & 4 & 3 \\
2 & 3 & 3 & 2
\end{matrix}
\]

The probabilities will be proportional to these.\\
%The total sum is $52$, and because one particle starts in each cell, we can just multiply it by $16$. This gives 
%\[
%\frac{x \cdot 16}{52} = \frac{x \cdot 4}{13}
%\]
%for each \( x \) in the second row. So the expected number of particles in each cell is
%\[
%\frac{12}{13}, \quad \frac{20}{13}, \quad \frac{20}{13}, \quad \frac{12}{13},
%\]
% and $A\cdot B\cdot C\cdot D=\left(\frac{12}{13}\cdot\frac{20}{13}\right)^2=\frac{240}{169}\implies \boxed{409}.$

Now let's find the energies. Denote the energy of cell $i$ as $E_i$. Also, we will denote $\beta = \frac{1}{k_B T}$. 
%The probability of begin in that cell is

%\[
%\frac{e^{-\beta E_i}}{\sum_j{e^{-\beta E_j}}}
%\]

%This means that the ratio of probabilities of two states is:
By the Boltzmann distribution:
\[
\frac{p(E_1)}{p(E_2)} = e^{\beta(E_2 - E_1)}
\]
\[
E_2 - E_1 = k_B T \log{\frac{p(E_1)}{p(E_2)}}
\]

This means that the difference in energies is maximized for biggest ratio. Which gives:

\[
\Delta E_{max} = k_B T \log{\frac{4}{2}} = \boxed{5.9728 \cdot 10 ^ {-8}\mathrm{eV}.}
\]



\end{solution}