\begin{solution}

Ở đây, chúng ta có thể sử dụng thông tin từ thực tế rằng đối với vật liệu dẫn điện, điện trường gần bề mặt vuông góc với vật dẫn. Đầu tiên, ta sẽ tính phân bố điện tích của một đĩa từ công thức \eqref{ellipsoidCharge}, sau đó thực hiện các phép tính tương tự giữa điện tĩnh và hấp dẫn.

Cho $x^2+y^2 = r^2$, $a=b=R$ và $c\rightarrow 0$.
\begin{gather*}
    \frac{x^2}{a^2} + \frac{y^2}{b^2} + \frac{z^2}{c^2} = 1\\
    \frac{r^2}{R^2}+\frac{z^2}{c^2}=1\\
    \frac{z^2}{c^2} = 1 - \frac{r^2}{R^2} 
\end{gather*}
Bây giờ, $\sigma$ trở thành:
\begin{equation*}
    \sigma = \frac{q}{4\pi R^2c} {\left( \frac{r^2}{R^4}
    + \frac{1 - \frac{r^2}{R^2}}{c^2} \right)} ^ {-1 / 2}
\end{equation*}

Vì $c^{-2} \gg a^{-2}$ nên:
\begin{gather*}
    \sigma \approx \frac{q}{4\pi R^2c} {\left(\frac{1 - \frac{r^2}{R^2}}{c^2}
    \right)} ^ {-1 / 2} 
\end{gather*}

Điều này cho ta phân bố điện tích trên một đĩa tích điện:
\begin{equation*}
    \sigma = \frac{q}{4\pi R^2} {\left(1 - \frac{r^2}{R^2}
    \right)} ^ {-1 / 2} 
\end{equation*}

Thực tế, đây là phân bố điện tích trên một trong các mặt của đĩa. 
Vì ta để  $c \rightarrow 0$ nên phải nhân kết quả này với hai (bạn có thể kiểm tra tích phân sẽ chỉ cho một nửa điện tích với công thức trên).
\begin{equation*}
    \sigma = \frac{q}{2\pi R^2} {\left(1 - \frac{r^2}{R^2}
    \right)} ^ {-1 / 2} 
\end{equation*}

Điện trường sinh ra bởi điện tích bề mặt $\sigma$ gần bề mặt là
\[ \mathrm{E} = \frac{\sigma}{2\varepsilon_0} \]
Và trường hấp dẫn gần bề mặt (cũng dùng định luật Gauss):
\[ 2 \mathrm{\Gamma} \mathrm{dS}= 4 \pi \mathrm{G} \rho h \mathrm{dS}\]
\[ \mathrm{\Gamma} = 2 \pi \mathrm{G} \rho h\]
Nên $\rho h$ hoạt động tương tự như $\sigma$. ($\Gamma$ là gia tốc hấp dẫn)

\begin{equation}
    \rho = \rho_0 {\left(1 - \frac{r^2}{R^2} \right)} ^ {-1 / 2} 
    \label{densityDistribution}
\end{equation}

Bây giờ cho phần tính toán số,
% (1 - 1 / 9) ^ ( - 1 / 2) / (1 - 4 / 9) ^ ( - 1 / 2)
\begin{equation*}
    \frac{\rho_{\frac{R}{3}}}{\rho_{\frac{2R}{3}}} =
    \sqrt{\frac{1 - \frac{4}{9}}{1 - \frac{1}{9}}} = 0.7905694
\end{equation*}

\end{solution}